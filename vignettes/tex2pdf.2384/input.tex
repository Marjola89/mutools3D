\documentclass[]{article}
\usepackage{lmodern}
\usepackage{amssymb,amsmath}
\usepackage{ifxetex,ifluatex}
\usepackage{fixltx2e} % provides \textsubscript
\ifnum 0\ifxetex 1\fi\ifluatex 1\fi=0 % if pdftex
  \usepackage[T1]{fontenc}
  \usepackage[utf8]{inputenc}
\else % if luatex or xelatex
  \ifxetex
    \usepackage{mathspec}
  \else
    \usepackage{fontspec}
  \fi
  \defaultfontfeatures{Ligatures=TeX,Scale=MatchLowercase}
\fi
% use upquote if available, for straight quotes in verbatim environments
\IfFileExists{upquote.sty}{\usepackage{upquote}}{}
% use microtype if available
\IfFileExists{microtype.sty}{%
\usepackage{microtype}
\UseMicrotypeSet[protrusion]{basicmath} % disable protrusion for tt fonts
}{}
\usepackage[margin=1in]{geometry}
\usepackage{hyperref}
\hypersetup{unicode=true,
            pdftitle={An introduction to mass univariate analysis of three-dimensional phenotypes},
            pdfauthor={Carlo Biffi},
            pdfborder={0 0 0},
            breaklinks=true}
\urlstyle{same}  % don't use monospace font for urls
\usepackage{color}
\usepackage{fancyvrb}
\newcommand{\VerbBar}{|}
\newcommand{\VERB}{\Verb[commandchars=\\\{\}]}
\DefineVerbatimEnvironment{Highlighting}{Verbatim}{commandchars=\\\{\}}
% Add ',fontsize=\small' for more characters per line
\usepackage{framed}
\definecolor{shadecolor}{RGB}{248,248,248}
\newenvironment{Shaded}{\begin{snugshade}}{\end{snugshade}}
\newcommand{\KeywordTok}[1]{\textcolor[rgb]{0.13,0.29,0.53}{\textbf{{#1}}}}
\newcommand{\DataTypeTok}[1]{\textcolor[rgb]{0.13,0.29,0.53}{{#1}}}
\newcommand{\DecValTok}[1]{\textcolor[rgb]{0.00,0.00,0.81}{{#1}}}
\newcommand{\BaseNTok}[1]{\textcolor[rgb]{0.00,0.00,0.81}{{#1}}}
\newcommand{\FloatTok}[1]{\textcolor[rgb]{0.00,0.00,0.81}{{#1}}}
\newcommand{\ConstantTok}[1]{\textcolor[rgb]{0.00,0.00,0.00}{{#1}}}
\newcommand{\CharTok}[1]{\textcolor[rgb]{0.31,0.60,0.02}{{#1}}}
\newcommand{\SpecialCharTok}[1]{\textcolor[rgb]{0.00,0.00,0.00}{{#1}}}
\newcommand{\StringTok}[1]{\textcolor[rgb]{0.31,0.60,0.02}{{#1}}}
\newcommand{\VerbatimStringTok}[1]{\textcolor[rgb]{0.31,0.60,0.02}{{#1}}}
\newcommand{\SpecialStringTok}[1]{\textcolor[rgb]{0.31,0.60,0.02}{{#1}}}
\newcommand{\ImportTok}[1]{{#1}}
\newcommand{\CommentTok}[1]{\textcolor[rgb]{0.56,0.35,0.01}{\textit{{#1}}}}
\newcommand{\DocumentationTok}[1]{\textcolor[rgb]{0.56,0.35,0.01}{\textbf{\textit{{#1}}}}}
\newcommand{\AnnotationTok}[1]{\textcolor[rgb]{0.56,0.35,0.01}{\textbf{\textit{{#1}}}}}
\newcommand{\CommentVarTok}[1]{\textcolor[rgb]{0.56,0.35,0.01}{\textbf{\textit{{#1}}}}}
\newcommand{\OtherTok}[1]{\textcolor[rgb]{0.56,0.35,0.01}{{#1}}}
\newcommand{\FunctionTok}[1]{\textcolor[rgb]{0.00,0.00,0.00}{{#1}}}
\newcommand{\VariableTok}[1]{\textcolor[rgb]{0.00,0.00,0.00}{{#1}}}
\newcommand{\ControlFlowTok}[1]{\textcolor[rgb]{0.13,0.29,0.53}{\textbf{{#1}}}}
\newcommand{\OperatorTok}[1]{\textcolor[rgb]{0.81,0.36,0.00}{\textbf{{#1}}}}
\newcommand{\BuiltInTok}[1]{{#1}}
\newcommand{\ExtensionTok}[1]{{#1}}
\newcommand{\PreprocessorTok}[1]{\textcolor[rgb]{0.56,0.35,0.01}{\textit{{#1}}}}
\newcommand{\AttributeTok}[1]{\textcolor[rgb]{0.77,0.63,0.00}{{#1}}}
\newcommand{\RegionMarkerTok}[1]{{#1}}
\newcommand{\InformationTok}[1]{\textcolor[rgb]{0.56,0.35,0.01}{\textbf{\textit{{#1}}}}}
\newcommand{\WarningTok}[1]{\textcolor[rgb]{0.56,0.35,0.01}{\textbf{\textit{{#1}}}}}
\newcommand{\AlertTok}[1]{\textcolor[rgb]{0.94,0.16,0.16}{{#1}}}
\newcommand{\ErrorTok}[1]{\textcolor[rgb]{0.64,0.00,0.00}{\textbf{{#1}}}}
\newcommand{\NormalTok}[1]{{#1}}
\usepackage{graphicx,grffile}
\makeatletter
\def\maxwidth{\ifdim\Gin@nat@width>\linewidth\linewidth\else\Gin@nat@width\fi}
\def\maxheight{\ifdim\Gin@nat@height>\textheight\textheight\else\Gin@nat@height\fi}
\makeatother
% Scale images if necessary, so that they will not overflow the page
% margins by default, and it is still possible to overwrite the defaults
% using explicit options in \includegraphics[width, height, ...]{}
\setkeys{Gin}{width=\maxwidth,height=\maxheight,keepaspectratio}
\IfFileExists{parskip.sty}{%
\usepackage{parskip}
}{% else
\setlength{\parindent}{0pt}
\setlength{\parskip}{6pt plus 2pt minus 1pt}
}
\setlength{\emergencystretch}{3em}  % prevent overfull lines
\providecommand{\tightlist}{%
  \setlength{\itemsep}{0pt}\setlength{\parskip}{0pt}}
\setcounter{secnumdepth}{0}
% Redefines (sub)paragraphs to behave more like sections
\ifx\paragraph\undefined\else
\let\oldparagraph\paragraph
\renewcommand{\paragraph}[1]{\oldparagraph{#1}\mbox{}}
\fi
\ifx\subparagraph\undefined\else
\let\oldsubparagraph\subparagraph
\renewcommand{\subparagraph}[1]{\oldsubparagraph{#1}\mbox{}}
\fi

%%% Use protect on footnotes to avoid problems with footnotes in titles
\let\rmarkdownfootnote\footnote%
\def\footnote{\protect\rmarkdownfootnote}

%%% Change title format to be more compact
\usepackage{titling}

% Create subtitle command for use in maketitle
\newcommand{\subtitle}[1]{
  \posttitle{
    \begin{center}\large#1\end{center}
    }
}

\setlength{\droptitle}{-2em}
  \title{An introduction to mass univariate analysis of three-dimensional
phenotypes}
  \pretitle{\vspace{\droptitle}\centering\huge}
  \posttitle{\par}
  \author{Carlo Biffi}
  \preauthor{\centering\large\emph}
  \postauthor{\par}
  \predate{\centering\large\emph}
  \postdate{\par}
  \date{2017-04-25}


\begin{document}
\maketitle

The R tools provided by \texttt{mutools3d} package enable to derive
association maps between clinical and genetic variables and
three-dimensional (3D) phenotypes defined on a triangular mesh. As an
examplar application, we provide data for the study of the association
between a synthetic clinical variable and wall thickness defined on a 3D
left ventricular mesh (atlas). In this tutorial, we are going to
illustrate:

\begin{enumerate}
\def\labelenumi{\arabic{enumi}.}
\tightlist
\item
  how to derive the regression coefficient at each vertex of the atlas
  representing the association between WT and the clinical variable
  under study (mass univariate regression).
\item
  how to boost belief in extended areas of signal by using
  threshold-free cluster enhacement (TFCE) together with an appropriate
  permutation strategy (Freedman-Lane procedure) to derive new p-values.
\end{enumerate}

\subsection{Mass Univariate
Regression}\label{mass-univariate-regression}

Mass univariate regression consists of applying a general linear model
\(Y = X\beta + \epsilon\) at each atlas vertex. The package provides
functions \texttt{mur} and \texttt{murHC4m} to perform mass univariate
regression, with the second function also applying HC4m heteroscedascity
consistent estimators to correct for violation of homoscedasticity
linear regression assumption. Input are a matrix \texttt{X} containig
the clinical variables to study for each subject and a matrix \texttt{Y}
containing at different columns the values at different vertices of a
three-dimensional phenotype - in this case wall thickness.

To load the dataset included in the package, please run:

\begin{Shaded}
\begin{Highlighting}[]
\KeywordTok{library}\NormalTok{(mutools3D)}
\KeywordTok{data}\NormalTok{(Xtest)}
\KeywordTok{data}\NormalTok{(Ytest)}
\end{Highlighting}
\end{Shaded}

\texttt{X} in this case is a {[}50x9{]} matrix containing the synthetic
clinical variable to study together with other covariates to adjust the
model and the intercept term, which must always be defined in the first
column. Categorical variables must be always coded using ``dummy''
coding. For example, in this case the variable ethnicity which had four
levels has been coded by picking the level 1 as a reference level and by
creating three dichotomous variables, where each variable contrasts with
level 1. \texttt{Y} is a {[}50x27623{]} matrix containing the values of
a three-dimensional phenotype at all the vertices under study.

The output of the mass univariate functions is a matrix storing the
regression coefficient \(\beta\), the t-statistic and the p-values
computed at each vertex and for each variable specified.

\begin{Shaded}
\begin{Highlighting}[]
\CommentTok{#extract results for synthetic variable}
\NormalTok{extract =}\StringTok{ }\DecValTok{6}

\CommentTok{#it is also possible to study more than one variable at the same time}
\CommentTok{#extract = c(5,6)}

\NormalTok{result =}\StringTok{ }\KeywordTok{mur}\NormalTok{(X,Y, extract)}
\CommentTok{#or}
\NormalTok{result =}\StringTok{ }\KeywordTok{murHC4m}\NormalTok{(X,Y, extract)}
\end{Highlighting}
\end{Shaded}

\subsection{Threshold-free cluster
enhancement}\label{threshold-free-cluster-enhancement}

Threshold-free cluster enhancement (TFCE) is an image-enhancement
technique that computes at each vertex of the triangular mesh a score
based on the extent and magnitude of the area of coherent signal that
surrounds it (Smith and Nichols 2009). Given a statistical map on a 3D
mesh, the TFCE function computes the corresponding TFCE map. In the
code, the 3D mesh is represented as a computational graph using a
two-columns matrix containing the mesh edges definitions: the first
column contains the ID of one vertex, the second column contains the ID
of the second vertex. To load an example of this matrix together with a
V-dimensional vector (V = number of vertices in the mesh) storing the
area associated with each vertex in the mesh, please run:

\begin{Shaded}
\begin{Highlighting}[]
\CommentTok{#load data for TFCE}
\KeywordTok{data}\NormalTok{(NNmatrix)}
\KeywordTok{data}\NormalTok{(areas)}
\end{Highlighting}
\end{Shaded}

TFCE can be then executed by using following code

\begin{Shaded}
\begin{Highlighting}[]
\CommentTok{#run TFCE on the t-statistic map previously obtained}
\NormalTok{TFCEresults =}\StringTok{ }\KeywordTok{TFCE}\NormalTok{(result[,}\DecValTok{2}\NormalTok{], A, NNmatrix)}
\end{Highlighting}
\end{Shaded}

TFCE-derived pvalues can be computed by performing permutation testing
on the input data. We developed a specific function to do so and that
implements the Freedman-Lane procedure for permutation testing of the
general linear model (Winkler and others 2014).

\begin{verbatim}
#compute TFCE-derived pvalues using Freedman-Lane procedure.

nPermutations = 1000
HC4m = TRUE
parallel = TRUE
nCores = 8
verbOutput = 0

TFCEresults = permFL(X, Y, extract, A, NNmatrix, nPermutations, HC4m, parallel, nCores, verbOutput)
\end{verbatim}

If \texttt{verbOutput} is set to 0 the output is a matrix containing in
its rows the pvalues computed at each vertex while the number of columns
referes to the variables specified in extract. If \texttt{verbOutput} is
set to 1 the output is a list where the \texttt{pval} field contains the
pvalues computed at each vertex, \texttt{TFCEmatrix} field contains a V
x nPermutations matrix containing the TFCE scores computed for each
permutation and the tfceScores field is a V-dimensional vector
containing the TFCE scores of the non-permuted data.

Last but not least, the derived pvalues using either TFCE or the
standard mass univariate approach need to be corrected for multiple
testing as we are performing statistical hypothesis testing at each
atlas vertex. For doing so, we suggest to use
\href{http://bioconductor.org/packages/release/bioc/html/multtest.html}{multtest
package}.

\subsection*{REFERENCES}\label{references}
\addcontentsline{toc}{subsection}{REFERENCES}

\hypertarget{refs}{}
\hypertarget{ref-smith2009}{}
Smith, Stephen M, and Thomas E Nichols. 2009. ``Threshold-Free Cluster
Enhancement: Addressing Problems of Smoothing, Threshold Dependence and
Localisation in Cluster Inference.'' \emph{NeuroImage} 44 (1). Elsevier:
83--98.

\hypertarget{ref-winkler2014}{}
Winkler, Anderson M, and others. 2014. ``Permutation Inference for the
General Linear Model.'' \emph{NeuroImage} 92. Elsevier: 381--97.


\end{document}
